% -- 第4章作業記録(MacOSX) --
\documentclass[a4j,titlepage]{jarticle}
\usepackage[dvipdfmx]{graphicx}
\usepackage{url}
\usepackage{epsfig}
\usepackage{ascmac}
%\usepackage{here}

\begin{document}

\subsection{動作確認}
作成した html ファイル(index, inside, basic)をブラウザ上で表示できるかの確認を行う.
Basic については,http, https どちらも表示できるか確認する.

% 0706
\subsection{MacOS 動作確認}
そのままでは HTML を閲覧することができないため,Safri の環境設定からプロキシ設定の変更を行う.\\
Safri の環境設定 → 詳細 → 詳細を変更 → 「Web プロキシ(HTTP)」と「保護された Web プロキシ(HTTPS)」のチェック項目を外す.\\
この作業により,閲覧が可能になり,自グループの HTML 文が表示される.

% 0710 
\subsection{MacOS 証明書の承認}
\begin{enumerate}
\item まず,以下よりルート認証局の証明書とし cacert.der ファイルをインストール
  \begin{center}
    \begin{screen}
\begin{verbatim}
  http://www.g9.info.kochi-tech.ac.jp/cacert.der
\end{verbatim}
    \end{screen}
  \end{center}

\item 証明書の追加を問われるため,追加を選択

\item 証明書をこれ以降信頼するように設定するか問われるため,「常に信頼」を選択し,パスワードとして root00 を入力

\item グループホームページである
  \begin{center}
    \begin{screen}
\begin{verbatim}
  https://www.g9.info.kochi-tech.ac.jp
\end{verbatim}
    \end{screen}
  \end{center}
にアクセスできるか確認し,表示できれば終了
  
\end{enumerate}

\end{document}
