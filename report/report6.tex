\documentclass[a4j,titlepage]{jarticle}
\usepackage[dvipdfmx]{graphicx}
\usepackage{url}
\usepackage{epsfig}
\usepackage{ascmac}
\usepackage{here}
%
\title{情報学群実験第4C 第6章レポート\\ ネットワークセキュリティ}
\author{学籍番号 1190358\\
        畠山 友華\\
        グループ9}
\date{\today}
%% プリアンブルここまで

%% 本文
\begin{document}
\maketitle
\section{目的}
% 背景からの実験の目的を記述
% --- 背景 ---
ルーティングを行い,TCP/IP の接続を向上させ DNS・Web サーバの構築することで,インターネットを通して アプリケーションの利用・Web ブラウザの閲覧・コンテンツの利用などが可能になった.これらが可能になったことで情報のやり取りにおける利便性が向上した.また,インターネットへの接続によって情報の取得が容易になったと同時に,不特定多数の利用者と管理者がいることでネットワーク上のなりすまし・不正アクセス・破壊などといった脅威にも晒される.これは,ルータを介して DNS で IP アドレスを用いて通信を行うためである.このような脅威に晒されてしまうと,膨大なデータを管理している企業やその他学校や組合などの組織において,多大な損害となってしまう.\\
% --- 目的 ---
ネットワークサービスを構築する際にも,ウィルス対策ソフトの導入や,OS のアップデート,サービスごとのアクセス制限などのコンピュータごとにセキュリティ設定を行うこともできる.しかしこれだけではセキュリティレベルはまだ低く,企業や学校と言った組織の運用を行う上で全ての端末において常に万全の状態で対策を行うことが必要となってくる.セキュリティレベルを万全のものにするために,1つのウィルスや何らかの脅威から LAN 内ネットワークで繋がれた端末全体にNAT(Network Address Translator) とは,ローカルなネッ危険が誘発される恐れがあるため,多重保護の観点からネットワーク単位での細かいセキュリティも重要となってくる.そのため,OS のアップデートやセキュリティ設定に加えてネットワークの入り口からの攻撃を防ぐように制御設定するのを目的とする.

\section{内容}
% 2つの実験の行う内容
ネットワークを通して情報のやり取りが可能になった現在,安全性を高めるために接続している LAN においてネットワークの入り口からセキュリティを強化し,内部ネットワークのセキュリティ確保を行う.\\
 まず,ネットワークセキュリティの制御設定を行うために,ネットワークのアクセスを選別し,必要ない通信を遮断する機能を持つファイアウォールを有効なものにする.このとき,ファイアウォールを構築するだけでは全ての攻撃を防ぐことはできない.そのため,ファイアウォールを運用するにあたって必ず他のセキュリティ技術と併用する必要がある.そこで,1つのプライベート IP アドレスから他のグローバル IP アドレスへと相互対応し変換して用いることのできる NAPT の設定,NAPT でできない外部ネットワークからの直接の通信を可能にするためのポートフォワーディングの設定も同時に行っていく.


\section{要素技術}
ここでは,ネットワークを安全に利用するために重要な役割を担っているセキュリティ技術としての「アプリケーションゲートウェイ」,「パケットフィルタ」,「NAT」,「ポートフォワーディング」について解説・説明を行っていく.

\subsection{アプリケーションゲートウェイ}
アプリケーションゲートウェイとは、ファイアウォールの一種でアプリケーション層でパケットの中継処理を行う,安全性の高いフィルタリング方式である.IP そうでパケットの中継処理を行うパケットフィルタリング方式とは違い,外部ネットワークと内部ネットワークを切り離した状態で,パケットの中継とアクセス制限を行う.アプリケーションゲートウェイは,そのままではインターネットに接続することのできない内部ネットワーク内のクライアントPC の代理としてインターンネット接続を行うサーバであるプロキシサーバを仲介することで,プロトコルごとの細かなアクセス制限が可能となる.\\
 アプリケーションゲートウェイの仕組みとしては,Web サーバからクライアントの要求に対するデータをクライアントに送信する.その際に,そのデータに対応したアプリケーションのプロキシサーバがデータを受け取り,問題がなければそのままクライアントへと送信される.しかし,問題が合った場合には,同じようにデータを送り過去のログ情報に基づいてチェックを行う.ここで問題が見つかると,クライアント PC に警告が表示され,問題の合ったデータはプロキシサーバで破棄される.この仕組みによって内部ネットワークへの危険の侵入を防ぐことができる.また,通過するパケッと の IP アドレスについては,NAT 機能により,内部ネットワークを隠蔽させたセキュリティ強度の高いフィルタリング制御を行っていく.\\
 セキュリティ強度を高めるために,アプリケーションゲートウェイを導入すると,細かな制御設定を行うため
処理速度が遅くなるといったデメリットがある.このため,アクセス量の多いネットワーク環境には処理速度が速いサーバシステムが必要になる\cite{bib:aptext}.

  \begin{figure}[htbp]
    \begin{center}
      \resizebox{120mm}{!}{\includegraphics{apuri.png}}
      \caption{アプリケーションゲートウェイ アクセス制御の仕組み}
     \label{fig:apuri}
   \end{center}
  \end{figure}

  また,アプリケーションゲートウェイのメリット・デメリットは以下のようになる.

  \begin{itemize}
  \item メリット
    \begin{itemize}
    \item 詳細なアクセス制限

    \item 詳細なログ情報の取得

    \item セキュリティ強度が高い

    \end{itemize}

  \item デメリット

    \begin{itemize}
    \item 処理速度が遅い

    \item アプリケーションごとに設定を変更する必要がある

   \end{itemize}

\subsection{パケットフィルタ}
ファイアウォールは,スタンドアロンな装置として構成することもできるが,ほとんどのファイアウォールは,スイッチやルータ内に組み込まれている.どちらの場合も,ファイアウォールを構築するのに使用される基盤となる機構は,パケットフィルタ(packet filter)として知られている.フィルタは,設定可能な機構から構成されていて,各パケットヘッダの中のフィールドを調べる.そして,そのパケットにルータの通過を許可するか破棄するを決定する.システム管理者は,どのどのパケットを通過させるかを指定して,パケットを設定する必要がある\cite{bib:filtertext}.\\
 パケットフィルタは以下の情報に基づいてデータ転送を制御(許可あるいは禁止)する.

\begin{itemize}
\item 推測されるデータの発信元のアドレス
\item データの送信先のアドレス
\item データを転送するために用いられるセッション及びアプリケーションのプロコル
\end{itemize}

 また,Cisco ルータを用いている場合において,ALC(アクセス・コントロール・リスト)というプロトコルや宛先 IP アドレスの指定から許可あるいは禁止,プロコル名,制御設定を行う.パケットフィルタの欠点として IP アドレス偽装に弱いことと,データのパケット部分を読み取るためデータの内容情報を検査することが不可能であるということが特徴としていえる.


\subsection{NAT}
NAT(Network Address Translator) とは,ローカルなネットワークでプライベート IP アドレスを使用し,インターネットへ接続するときにグローバル IP アドレスへ変換する技術として開発された.さらに,アドレスだけでなく TCP や UDP のポート番号も付け替える NAPT(Network Address Ports Translator) が登場したことで,1つのグローバル IP アドレスで複数のホスト間の通信が可能になった.\\
 NAT(NAPT)は,基本的にはアドレスが枯渇している IPv4 のために生まれた技術である.ただし, IPv6 でもセキュリティの向上のために NAT が利用されたり, IPv4 と IPv6 の間で相互通信をするための技術として利用される.\\

\subsubsection{NAT の仕組み}
 図\ref{fig:nat}のように,10.0.0.10 のホストが 163.221.120.9 と通信する場合について考えてみると,NAT を利用することで,途中の NAT ルータで送信元 IP アドレスである 10.0.0.10 がグローバル IP アドレス 202.244.174.37 に変換されてから転送される.逆に, 163.221.120.9 からパケットが送られてきた時には,宛先 IP アドレス 202.244.174.37 がプライベート IP アドレス 202.244.174.37 がプライベート IP アドレス 10.0 0.10 として変換されて転送される仕組みになっている.

  \begin{figure}[htbp]
    \begin{center}
      \resizebox{120mm}{!}{\includegraphics{nat.png}}
      \caption{NATの仕組み}
      \label{fig:nat}

   \end{center}
  \end{figure}

\subsubsection{NAPT の仕組み}
 NAT や NAPT 対応ルータの内部ではアドレス変換のためのテーブルが自動的に作られることが多い.その際に,10.0.0.10 から 163.221.120.9 に向けて最初にパケットが送信されたときに変換テーブルが作られ,それに対応した処理が行われる.\\
 IP アドレスを変換するだけでは,プライベートネットワーク内の多数のホストから通信が行われる場合には,変換先のグローバル IP アドレスが不足する事態を解決するために図\ref{fig:napt}のようにポート番号も含めて変換処理を行う.これが NAPT の特徴となっている.\\
 具体的には,図\ref{fig:napt}では,163.221.120.9 のホストのポート 80番に,10.0.0.10 と 10.0.0.11 の2つのホストから送信元ポート番号が同じ 1025番で接続している.しかし,IP アドレスをグローバル IP アドレスに変換しただけでは識別に必要な数字が全て同じになってしまう.そのた,このときの 10.0.0.11 の送信元ポート番号の 1025 を 1026 に帰ることで通信を識別できるようにすることが可能となる.図\ref{fig:napt}のようになテーブルが NAPT 対応ルータに作られていれば,パケットを送受信したときに正しく変換することができ,クライアントA,クライアントB とサーバの間で同時に通信できるようになる.\\
 このような変換テーブルは NAT 対応ルータによって自動的に更新される.例えば,TCP の場合には,TCP のコネクション確立を意味する SYN パケットが流れたときにテーブルが作られ,FIN パケットが流れてその確認応答がされた後でテーブルから消去される\cite{bib:iptext}.


  \begin{figure}[htbp]
    \begin{center}
      \resizebox{120mm}{!}{\includegraphics{napt.png}}
      \caption{NAPTの仕組み}
     \label{fig:napt}
   \end{center}
  \end{figure}


\subsection{ポートフォワーディング}
 NAT を行うと,内側から発信されたパケットの返信ではなく,外側ネットワークから直接外側 IP アドレス宛に行われる通信は,NAT てーぶるにエントリーがないため,転送することができない.これより,公開洋サーバを内側に置くことができないという問題点がある.\\
 そこで,ポートフォワーディングは,特定の宛先ポート番号のパケットを受け取ると,宛先 IP アドレスだけを変換して社内の公開サーバへパケットを転送する機能がある.これは,静的 IP マスカレードとも呼ばれる.例えば,宛先ポート番号が 80番のパケットを受信したルータは,社外のクライアントに代わって,社内の Web サーバへとアクセスする.こうして,外部に対してはあたかもルータが公開 Web サーバであるかのように見せかけることができるようになっている.

\section{作業記録}
ここでは,セキュリティ対策として制御設定し安全に通信を行うために,ファイアウォールを以下の手順に沿って設定していく.また,ファイアウォールだけでは不完全となるため,NAPT やポートフォワーディングも設定していく.

\subsection{アクセスリストによる設定}
\begin{itemize}
\item Windows PCとルータをシリアルケーブルで接続し,Puttyより設定行う.

  \item 以下のアクセスリスト項目を入力して通信制御設定を行う.
\begin{screen}
    \begin{center}
\begin{verbatim}
no access-list 100

access-list 100 permit udp any host 172.21.19.2 eq 53
access-list 100 permit tcp any host 172.21.19.2 eq 53

access-list 100 permit tcp any host 172.21.19.2 eq 25

access-list 100 permit tcp any host 172.21.19.2 eq 80
access-list 100 permit tcp any host 172.21.19.2 eq 443

access-list 100 permit ip any host 192.168.0.19
access-list 100 permit ip any host 192.168.0.255
access-list 100 permit ip any host 255.255.255.255

access-list 100 permit ip any host 224.0.0.5
access-list 100 permit ip any host 224.0.0.6

access-list 100 permit udp any eq 53 host 172.21.19.2
access-list 100 permit tcp any eq 53 host 172.21.19.2
access-list 100 permit tcp any eq 25 host 172.21.19.2
access-list 100 permit tcp any eq 80 host 172.21.19.2
access-list 100 permit tcp any eq 443 host 172.21.19.2

no access-list 101
access-list 101 permit udp host 172.21.19.2 any eq 53
access-list 101 permit tcp host 172.21.19.2 any eq 53
access-list 101 permit tcp host 172.21.19.2 any eq 25
access-list 101 permit tcp host 172.21.19.2 any eq 80
access-list 101 permit tcp host 172.21.19.2 any eq 443 
access-list 101 permit udp host 172.21.19.2 eq 53 any
access-list 101 permit tcp host 172.21.19.2 eq 53 any
access-list 101 permit tcp host 172.21.19.2 eq 25 any
access-list 101 permit tcp host 172.21.19.2 eq 80 any
access-list 101 permit tcp host 172.21.19.2 eq 443 any
\end{verbatim}
    \end{center}
  \end{screen}

\item Puttyを開いてメモ帳に記した上記の内容をコピーして記述する.

\item ルータでインターフェースへの適用を行う.
  \begin{screen}
    \begin{center}
\begin{verbatim}
% en
# conf term
# interface FastEthernet0
# io access-group 100 in
# end

# interface FastEthernet1
# io access-group 101 in
# end
\end{verbatim}
    \end{center}
  \end{screen}

\subsection{NAPT の設定}
  
\item ここで,NAPT の設定を行うために,アクセスリストの除去を行う.
  
\begin{screen}
    \begin{center}
\begin{verbatim}
% en
# conf term
# no access-list 100
# no access-list 101
\end{verbatim}
    \end{center}
\end{screen}

\item 以下の文によってプールの設定を行う.
   \begin{screen}
    \begin{center}
\begin{verbatim}
# ip nat pool ovrld 172.16.10.1 172.16.10.1 prefix 24
\end{verbatim}
    \end{center}
   \end{screen}

\item 以下の記述によって NAPT の設定ができる.
   \begin{screen}
    \begin{center}
\begin{verbatim}
# ip nat inside source list 7 pool ovrld overload
\end{verbatim}
    \end{center}
   \end{screen}
 
\item 以下のとおりに入力して設定を行う.
  
   \begin{screen}
    \begin{center}
\begin{verbatim}
# interface fastethernet 0
# ip nat outside
# exit

# interface fastethernet 1
# ip nat inside
# exit
\end{verbatim}
    \end{center}
   \end{screen}

   \item 他のネットワークから自分のグループのサーバにアクセスできるようにポートフォワーディングの設定を行なった.

    \begin{screen}
    \begin{center}
\begin{verbatim}
# ip nat inside source static tcp 172.21.14.2 80 192.168.0.14 80
# ip nat inside source static tcp 172.21.14.2 25 192.168.0.14 25
# ip nat inside source static tcp 172.21.14.2 53 192.168.0.14 53
# ip nat inside source static udp 172.21.14.2 53 192.168.0.14 53
\end{verbatim}
    \end{center}
    \end{screen}

\subsection{サーバのIPアドレス情報の変更}
NATを定義して,内部ネットワークのコンピュータはルータの外側IPアドレスによってインターネットと接続するようにしても,DNSにおいて,サーバコンピュータのドメイン名に対応したIPアドレスは,内部ネットワークでのサーバコンピュータのもののままである.よって,サーバコンピュータで管理者権限となった後,ゾーンファイル/etc/bind/g9.zoneをviエディタで開き,9行目を以下のように変更する.ここでは,サーバコンピュータのドメイン名に対応したIPアドレスをルータの外側IPアドレスにしている.
\begin{screen}
\begin{center}
\begin{verbatim}
server  IN      A       192.168.0.19
\end{verbatim}
\end{center}
\end{screen}


\section{考察}
 インターネットを安全に安心して利用することができるために,ネットワークセキュリティが存在し,このセキュリティ対策を行うことで様々な悪質なインターンネットを通した脅威や危険から防ぐことができる.ネットワークセキュリティには,様々な通信に対して対策技術のメリットやデメリットからそれぞれに適したセキュリティ技術が用いられている.\\
 今回設定したファイアウォールでも,使用するものによってセキュリティの対応も変わってくるし,それぞれに対応した技術を導入し,また設定していく必要がある.この技術は,1つ設定しただけでは完全とは言えず,いくつか組み合わせることで発揮したり安心できるものへとなっていく.組み合わせに関しても適宜許可したり,拒否したりと設定も増えてしまう事になる.これは,面倒であったり,一般ユーザにとっては設定方法がわからなかったり,このぐらい大丈夫だろうと油断してしまう可能性も高い.このことから,コンピュータに対してセキュリティが弱まっていると警告するだけでなく,それに対してどの動作をすると解決されるのか提案したり設定次第で適当な対策を自動的に制御設定を行ってくれるような機能があれば利用者側としても安心して利用する事ができるのではないかと考える.この技術としては,今注目されている機械学習などの技術の応用として対策することはできないかと考える.しかし,ただセキュリティ対策の提案をするだけでは信ぴょう性が低く,似たようなウィルスもあるためアプリケーションを作成してアプリケーション上で行うことで信憑性が高い対策技術へと変えることができると考える.

\begin{thebibliography}{9}
 \bibitem{bib:aptext}
    セキュリティ研究会 著,
    ``まるごと図解 最新 インターネットセキュリティがわかる,''
    株式会社技術評論社,2000
\bibitem{bib:pakettext}
    D.Brent Chapman(D. ブレント・チャップマン), Elizabeth D.Zwicky(エリザベス・D.ツイッキー) 著,
    ``ファイアウォール構築 インターネット・セキュテティ,''
    株式会社オーラリー・ジャパン,1996
\bibitem{bib:iptext}
    竹下隆史・村山公保・荒井透・苅田幸雄 著,
    ``マスタリング TCP/IP 入門編 第5版,''
    株式会社 オーム社,2004.
\bibitem{bib:filtertext}
    Douglas E.Comer 著,
    ``コンピュータネットワークとインターネット:始原から現代まで, 深化する技術とその基礎理論,''
    翔泳社,2015
\bibitem{bib:ssltext}
    ICT ワークショップ 著,
    ``情報処理教科書 ネットワークスペシャリスト 2017年版,''
    翔泳社,2017 
\end{thebibliography}

\end{document}
