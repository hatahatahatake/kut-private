% -- 第3章作業記録(Windows7) --
\documentclass[a4j,titlepage]{jarticle}
\usepackage[dvipdfmx]{graphicx}
\usepackage{url}
\usepackage{epsfig}
\usepackage{ascmac}
%\usepackage{here}

\begin{document}

\subsection{Windows7のリゾルバ設定}
Windows7 では,「コントールパネル」→「ネットワークとインターネット」→「ネットワークと共有センター」→「アダプターの設定」と順に選択し,用いているネットワークインターフェースのプロパティから,IPv4 の設定の項の DNS サーバを変更し,リゾルバの設定とする.

\subsection{Windows7のMUAの設定}
\begin{itemize}
\item まず,以下をインストールし, Japanese を選択して Mozilla Thunderbird のセットアップを開始する.
  \begin{center}
    \begin{screen}
\begin{verbatim}
 https://www.mozilla.org/en-US/thunderbird/all 
\end{verbatim}
    \end{screen}
  \end{center}

\item セットアップが完了すると新しいメールアカウント設定に進む.ここで,
  \begin{center}
    \begin{screen}
\begin{verbatim}
  あなたのお名前:適当でいいがローマ字が無難
  メールアドレス:ユーザ@gX.info.kochi-tech.ac.jp
  パスワード:サーバで設定したパスワード
\end{verbatim}
    \end{screen}
  \end{center}

 これらを入力し,設定する(Xはグループ番号).

\item 入力が完了すると,以下の項目が表示される.
  \begin{center}
    \begin{screen}
\begin{verbatim}
 受信サーバ: pop3, pop.gXinfo.kochi-tech.ac.jp
 送信サーバ: SMTP, smtp.gXinfo.kochi-tech.ac.jp
 ユーザ名:設定したユーザ名
\end{verbatim}
    \end{screen}
  \end{center}

 これらが合っているか確認をし,メールを送信して送信できたら完了となる.

\end{itemize}

\end{document}
